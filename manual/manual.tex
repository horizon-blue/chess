\documentclass[12pt]{article}
\usepackage[letterpaper, margin=1in]{geometry}
\usepackage{graphicx}
\graphicspath{ {img/} }

\title{Chess Game GUI Testing Manual}
\author{Xiaoyan Wang (xiaoyan5@illinois.edu)}

\begin{document}
\maketitle

\section{Game Board}

It is better if we can check the functionality of game board before
we check how individual pieces interact with each other. My Chess board
is implemented to support re-sizing - so we could start here.

\subsection{Empty Board}

Here is how the empty board should look like under several different
sizes:

\begin{figure}[!h]
\begin{center}
\includegraphics[scale=0.45]{empty-8x8}
\caption{An empty $8\times 8$ board}
\end{center}
\end{figure}
\begin{figure}[!h]
\begin{center}
\includegraphics[scale=0.45]{empty-4x7}
\caption{An empty $4\times 7$ board}
\end{center}
\end{figure}
\begin{figure}[!h]
\begin{center}
\includegraphics[scale=0.45]{empty-12x4}
\caption{An empty $12\times 4$ board}
\end{center}
\end{figure}

\subsection{Filled Board}

After initializing the board, you should be able to see pieces appears on
the GUI. Make sure that the pieces are on the correct positions and
have the correct Z-Index:
\begin{figure}[!h]
\begin{center}
\includegraphics[scale=0.45]{full-8x8}
\caption{A full $8\times 8$ board}
\end{center}
\end{figure}
\begin{figure}[!h]
\begin{center}
\includegraphics[scale=0.45]{full-6x6}
\caption{A full $6\times 6$ board}
\end{center}
\end{figure}
\begin{figure}[!h]
\begin{center}
\includegraphics[scale=0.40]{full-10x10}
\caption{A full $10\times 10$ board}
\end{center}
\end{figure}



\end{document}